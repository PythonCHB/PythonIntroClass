\documentclass{beamer}
%\usepackage[latin1]{inputenc}
\usetheme{Warsaw}
\title[Intro to Python: Week 2]{Introduction  to Python\\ More functions, ...}
\author{Christopher Barker}
\institute{UW Continuing Education / Isilon}
\date{June 27, 2012}

\usepackage{listings}
\usepackage{hyperref}

\begin{document}

% ---------------------------------------------
\begin{frame}
  \titlepage
\end{frame}

% ---------------------------------------------
\begin{frame}
\frametitle{Table of Contents}
%\tableofcontents[currentsection]
  \tableofcontents
\end{frame}


\section{Review/Questions}

% ---------------------------------------------
\begin{frame}{Review of Previous Class}

\begin{itemize}
  \item ...
\end{itemize}

\end{frame}


% header
% ---------------------------------------------
\begin{frame}{Homework review}

  {\Large Homework notes }

\end{frame}

\section{More on functions}

% ---------------------------------------------
\begin{frame}[fragile]{Default Parameters}

 {\Large Sometimes you don't need the user to specify everything every time}

\begin{verbatim}
def fun(x,y,z=5):
    print x,y,z
\end{verbatim}

\end{frame} 


\section{More on Strings}

\section{More with strings}

% ---------------------------------------------
\begin{frame}[fragile]{Building Strings}

\begin{verbatim}
In [178]: "a string"
Out[178]: 'a string'

In [179]: str(34.5)
Out[179]: '34.5'

In [180]: `34.56`
Out[180]: '34.56'

In [181]: "the number %s is %i"%('five', 5)
Out[181]: 'the number five is 5'
\end{verbatim}

\end{frame} 


% ---------------------------------------------
\begin{frame}{String methods}

  {\Large bunch of...}

\end{frame}

\begin{frame}{Sequence API}

  {\Large full API}

\url{http://docs.python.org/library/stdtypes.html#sequence-types-str-unicode-list-tuple-bytearray-buffer-xrange}

\end{frame}


\end{document}

 
