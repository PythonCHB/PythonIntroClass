\documentclass{beamer}
%\usepackage[latin1]{inputenc}
\usetheme{Warsaw}
\title[Intro to Python: Week 2]{Introduction  to Python\\ Modules, Conditionals, Looping}
\author{Christopher Barker}
\institute{UW Continuing Education / Isilon}
\date{June 27, 2012}

\usepackage{listings}
\usepackage{hyperref}

\begin{document}

% ---------------------------------------------
\begin{frame}
  \titlepage
\end{frame}

% ---------------------------------------------
\begin{frame}
\frametitle{Table of Contents}
%\tableofcontents[currentsection]
  \tableofcontents
\end{frame}


\section{Review/Questions}

% ---------------------------------------------
\begin{frame}{Review of Previous Class}

\begin{itemize}
  \item Values and Types
  \item Expressions
  \item Intro to functions
  \item Scopes and functions 
\end{itemize}

\end{frame}


% ---------------------------------------------
\begin{frame}{Lightning Talks}

\vfill
{\Large Lightning talks by Cliff and Myke today}

\vfill
{\large ( in that order )}
\vfill

\end{frame}


% ---------------------------------------------
\begin{frame}{Homework review}

  {\Large Homework Questions? }

\end{frame}

\begin{frame}[fragile]{Functions as Objects}

  {\Large Passing a function as an argument}

\begin{verbatim}
def simple():
    print "I'm simple"

def run_twice(fun):
    fun()
    fun()

run_twice( simple )
vs.
run_twice( simple() )
\end{verbatim}

\end{frame}


% ---------------------------------------------
\begin{frame}{Note about lectures}

{\LARGE
\vfill
Hopefully, you are reading the book(s)
\vfill
I'm not going to be comprehensive--
\vfill
I'll focus on overview and ``gotchas'' 
}
\end{frame}

% ---------------------------------------------
\begin{frame}{Notes on reading}

{\LARGE
\vfill
Checking for type -- factorial example
\vfill
tuple unpacking in function arguments -- \\[0.2in]
Pythagoras example
}
\end{frame}



%***********************************
\section{Python Code Structure}

% ---------------------------------------------
\begin{frame}[fragile]{Code Structure}

{\Large Python is all about namespaces --  the ``dots'' }

\vspace{0.2in}
\verb+name.another_name+
\vspace{0.2in}

the ``dot'' indicates looking for a name in the namespace of the given object.

could be:

\begin{itemize}
\item name in a module
\item module in a package
\item attribute of an object
\item method of an object
\end{itemize}

\end{frame}

% ---------------------------------------------
\begin{frame}[fragile]{indenting and blocks}

{\Large  Indenting determines blocks of code }

\vfill
\begin{verbatim}
something:
    some code
    some more code
    another block:
        code in 
        that block
\end{verbatim}

\vfill
{\Large But you need the colon too...}

\end{frame}

% ---------------------------------------------
\begin{frame}[fragile]{indenting and blocks}

{\Large  You can put a one-liner after the colon:}

\vfill
\begin{verbatim}
In [167]: x = 12

In [168]: if x > 4: print x
12
\end{verbatim}

\vfill
{\Large Only do this if it makes it more readable...}

\end{frame}


\begin{frame}{spaces and tabs}

{\Large  An indent can be:}
\begin{itemize}
  \item Any number of spaces
  \item A tab
  \item tabs and spaces:
    \begin{itemize}
      \item A tab is eight spaces (always!)
      \item Are they eight in your editor?
    \end{itemize}
\end{itemize}

\vfill
{\LARGE Use four spaces -- really!}

\vfill
(PEP 8)

\end{frame}


% ---------------------------------------------
\begin{frame}[fragile]{Spaces Elsewhere}

{\Large  Other than indenting -- space doesn't matter}

\vfill
\begin{verbatim}

x = 3*4+12/func(x,y,z)

x = 3*4 + 12 /   func (x,   y, z) 

\end{verbatim}

\vfill
{\Large Choose based on readability/coding style}

\vfill
\center{\LARGE PEP 8}

\end{frame}


% ---------------------------------------------
\begin{frame}[fragile]{Various Brackets}

{\Large Bracket types:}

\begin{itemize}
  \item parentheses \verb+( )+
    \begin{itemize}
      \item tuple literal: \verb+(1,2,3)+
      \item function call: \verb+fun( arg1, arg2 )+
      \item grouping: \verb| (a + b) * c |
    \end{itemize}
  \item square brackets \verb+[ ]+
    \begin{itemize}
      \item list literal: \verb+[1,2,3]+
      \item sequence indexing: \verb+a_string[4]+
    \end{itemize}
  \item curly brackets \verb+{ }+
    \begin{itemize}
      \item dictionary literal: \verb+{"this":3, "that":6}+
      \item (we'll get to those...)
    \end{itemize}
\end{itemize}

\end{frame}


% ---------------------------------------------
\begin{frame}[fragile]{tuples and commas..}

{\Large  Tuples don't NEED parentheses... }

\begin{verbatim}
In [161]: t = (1,2,3)
In [162]: t
Out[162]: (1, 2, 3)

In [163]: t = 1,2,3
In [164]: t
Out[164]: (1, 2, 3)

In [165]: type(t)
Out[165]: tuple
\end{verbatim}

\end{frame}

% ---------------------------------------------
\begin{frame}[fragile]{tuples and commas..}

{\Large  Tuples do need commas... }

\begin{verbatim}
In [156]: t = ( 3 )

In [157]: type(t)
Out[157]: int

In [158]: t = (3,)
In [159]: t
Out[159]: (3,)

In [160]: type(t)
Out[160]: tuple
\end{verbatim}

\end{frame}

%%-------------------------------
%\begin{frame}{LAB}
%
%\begin{itemize}
%  \item Something with tuples, etc, etc...
%\end{itemize}
%
%\end{frame}


\section{Modules and packages}

\begin{frame}{modules and packages}

{\Large A module is simply a namespace}

\vfill
{\Large A package is a module with other modules in it}

\vfill
{\Large The code in the module is run when it is imported}

\end{frame}

% ---------------------------------------------
\begin{frame}[fragile]{importing modules}

\begin{verbatim}

import modulename

from modulename import this, that

import modulename as a_new_name
\end{verbatim}
\vfill
(demo)
\end{frame} 

% ---------------------------------------------
\begin{frame}[fragile]{importing from packages}

\begin{verbatim}

import packagename.modulename

from packagename.modulename import this, that

from package import modulename

\end{verbatim}
\vfill
(demo)

\vfill
\url{http://effbot.org/zone/import-confusion.htm}

\end{frame} 

% ---------------------------------------------
\begin{frame}[fragile]{importing from packages}

\begin{verbatim}
from modulename import *
\end{verbatim}

\vfill
{\LARGE Don't do this!}
\vfill
{\Large (``Namespaces are one honking great idea...'')}

\vfill
(wxPython and numpy example...)

\vfill
Except \emph{maybe} math module

\vfill
(demo)
\end{frame} 


%------------------------------------
\begin{frame}[fragile]{import}

\vfill
If you don’t know the module name before execution.

\vfill
\begin{verbatim}
__import__(module)
\end{verbatim}

\vfill
where \verb+module+ is a Python string.

\vfill
\end{frame}

\begin{frame}[fragile]{modules and packages}

\vfill
{\Large The code in a module is NOT re-run when imported again
 -- it must be explicitly reloaded to be re-run}

\begin{verbatim}
import modulename

reload(modulename)
\end{verbatim}

(demo)

\begin{verbatim}
import sys
print sys.modules
\end{verbatim}
(demo)
\end{frame}


%-------------------------------
\begin{frame}[fragile]{LAB}

{\Large  Experiment with importing different ways:}
\begin{verbatim}
import math
dir(math) # or, in ipython -- math.<tab>
math.sqrt(4)

import math as m
m.sqrt(4)

from math import *
sqrt(4)
\end{verbatim}

\end{frame}

%-------------------------------
\begin{frame}[fragile]{LAB}

{\Large  Experiment with importing different ways:}
\begin{verbatim}
import sys
print sys.path

import os
print os.path
\end{verbatim}
{\Large You wouldn't want to import * those -- check out}
\begin{verbatim}
os.path.split()
os.path.join()
\end{verbatim}

\end{frame}

%-------------------------------
\begin{frame}{Lightning Talk}

{\LARGE Lightning Talk: Cliff  }
\end{frame}


\section{Boolean Expressions}

% ---------------------------------------------
\begin{frame}[fragile]{Truthiness}

{\Large What is true or false in Python?}

\begin{itemize}
  \item The Booleans: \verb+True+ and \verb+False+
  \item ``Something or Nothing''
\end{itemize}

{\small \url{http://mail.python.org/pipermail/python-dev/2002-April/022107.html} }

\end{frame}

%-------------------------------
\begin{frame}[fragile]{Truthiness}

{\Large determining Truthiness:

\vfill
  \verb+bool(something)+
\vfill
}

\end{frame}

% ---------------------------------------------
\begin{frame}[fragile]{Boolean Expressions}

{\Large \verb+False+ }

\begin{itemize}
  \item \verb+None+
  \item \verb+False+
  \item zero of any numeric type, for example, \verb+ 0, 0L, 0.0, 0j+.
  \item any empty sequence, for example, \verb+ '', (), [] +.
  \item any empty mapping, for example, \verb+{}+.
  \item instances of user-defined classes, if the class defines a
        \verb+__nonzero__() or __len__()+ method, when that method
        returns the integer zero or bool value \verb+False+.
\end{itemize}

\url{http://docs.python.org/library/stdtypes.html}

\end{frame}

% ---------------------------------------------
\begin{frame}[fragile]{Boolean Expressions}

{ \Largeavoid }

\vspace{0.25in}
\verb+if xx == True:+

\vspace{0.25in}
{ \Largeuse }

\vspace{0.25in}
\verb+if xx:+

\end{frame}

% ---------------------------------------------
\begin{frame}[fragile]{Boolean Expressions}

{\Large ``Shortcutting''}

\begin{verbatim}
                  if x is false, 
x or y               return y,
                     else return x

                  if x is false,
x and y               return  x
                      else return y

                  if x is false,
not x               return True,
                    else return False 
\end{verbatim}

\end{frame} 

% ---------------------------------------------
\begin{frame}[fragile]{Boolean Expressions}

{\Large Stringing them together}

\begin{verbatim}
 a or b or c or d

a and b and c and d  
\end{verbatim}

{\Large The first value that defines the result is returned}

\vfill
(demo)
\end{frame}


%---------------------------------------------
\begin{frame}[fragile]{Boolean returns}

{\Large From CodingBat}
\vfill
\begin{verbatim}
def sleep_in(weekday, vacation):
    if weekday == True and vacation == False:
        return False
    else:
        return True
\end{verbatim}

\end{frame}



%---------------------------------------------
\begin{frame}[fragile]{Boolean returns}

{\Large From CodingBat}

%\begin{verbatim}
%def makes10(a, b):
%    return a == 10 or b == 10 or a+b == 10
%\end{verbatim}

\begin{verbatim}
def sleep_in(weekday, vacation):
    return not (weekday == True and vacation == False)
\end{verbatim}

or

\begin{verbatim}
def sleep_in(weekday, vacation):
    return (not weekday) or vacation
\end{verbatim}


\end{frame}


% -------------------------------------------
\begin{frame}[fragile]{bools are ints?}

{\Large bool types are subclasses of integer}

\begin{verbatim}
In [1]: True == 1
Out[1]: True

In [2]: False == 0
Out[2]: True  
\end{verbatim}

{\Large It gets weirder! }

\begin{verbatim}
In [6]: 3 + True
Out[6]: 4
\end{verbatim}

(demo)

\end{frame}


%-------------------------------
\begin{frame}{LAB}

\vfill
{\large re-write a couple CodingBat exercises, returning the direct boolean results}
\vfill

\end{frame}

\section{Conditionals}

%-------------------------------
\begin{frame}[fragile]{if}

{\Large Making Decisions...}
\begin{verbatim}
if a:
    print 'a'
elif b:
    print 'b'
elif c:
    print 'c'
else:
    print 'that was unexpected'
\end{verbatim}

\end{frame}


%-------------------------------
\begin{frame}[fragile]{if}

{\Large Making Decisions...}
\begin{verbatim}
if a:
    print 'a'
elif b:
    print 'b'

## versus...

if a:
    print 'a'
if b:
    print 'b'
\end{verbatim}

\end{frame}

%-------------------------------
\begin{frame}[fragile]{switch?}

\vfill
{\Large No switch/case in Python}

\vfill
{\Large use use \verb|if..elif..elif..else|}

\vfill

(or a dictionary, or subclassing....)
\end{frame}

%%-------------------------------
%\begin{frame}[fragile]{LAB}
%
%\begin{itemize}
%
%
%\item some if .. elif, etc. examples
%
%\end{itemize}
%
%\end{frame}

% ************************************
\section {Sequences}

\begin{frame}[fragile]{Sequences}

{\Large Sequences are ordered collections}

\vfill
{\Large They can be indexed, sliced, iterated over,...}

\vfill
{\Large They have a length:  \verb+len(sequence)+}

\vfill
{\Large Common sequences (Duck Typing)}

\begin{itemize}
   \item strings
   \item tuples
   \item lists
\end{itemize}

\end{frame}

%-------------------------------
\begin{frame}[fragile]{Indexing}

{\Large square brackets for indexing: \verb+[]+}

\vfill
{\Large Indexing starts at zero}

\begin{verbatim}
In [98]: s = "this is a string"

In [99]: s[0]
Out[99]: 't'

In [100]: s[5]
Out[100]: 'i'
\end{verbatim}

\end{frame}

%-------------------------------
\begin{frame}[fragile]{Indexing}

{\Large Negative indexes count from the end}

\vfill
\begin{verbatim}
In [105]: s = "this is a string"

In [106]: s[-1]
Out[106]: 'g'

In [107]: s[-6]
Out[107]: 's'
\end{verbatim}

\end{frame}

%-------------------------------
\begin{frame}[fragile]{Slices}

{\Large Slicing: Pulling a range out of a sequence}

\begin{verbatim}
sequence[start:finish]  

indexes for which:

start <= i < finish
\end{verbatim}

\end{frame}

%-------------------------------
\begin{frame}[fragile]{Slices}
\begin{verbatim}
In [121]: s = "a bunch of words"
In [122]: s[2]
Out[122]: 'b'

In [123]: s[6]
Out[123]: 'h'

In [124]: s[2:6]
Out[124]: 'bunc'

In [125]: s[2:7]
Out[125]: 'bunch'
\end{verbatim}

\end{frame}


%-------------------------------
\begin{frame}[fragile]{Slices}

{\Large the indexes point to the spaces between the items}

\vfill
\begin{verbatim}
   X   X   X   X   X   X   X   X
 |   |   |   |   |   |   |   | 
 0   1   2   3   4   5   6   7
\end{verbatim}

\end{frame}

%-------------------------------
\begin{frame}[fragile]{Slices}

{\Large Slicing satisfies nifty properties:

\vfill
\begin{verbatim}
len( seq[a:b] ) == b - a

seq[a:b] + seq [b:c] == seq

\end{verbatim}

}

\end{frame}

% ------------------------------------------------
\begin{frame}[fragile]{Slicing vs. Indexing}

Indexing returns a single element

\begin{verbatim}
In [86]: l
Out[86]: [0, 1, 2, 3, 4, 5, 6, 7, 8, 9]

In [87]: type(l)
Out[87]: list

In [88]: l[3]
Out[88]: 3

In [89]: type( l[3] )
Out[89]: int
\end{verbatim}
\end{frame}

% ------------------------------------------------
\begin{frame}[fragile]{Slicing vs. Indexing}

Unless it's a string:

\begin{verbatim}
In [75]: s = "a string"

In [76]: s[3]
Out[76]: 't'

In [77]: type(s[3])
Out[77]: str
\end{verbatim}

there is no single character type

\end{frame}


%-------------------------------
\begin{frame}[fragile]{Slicing vs. Indexing}

Slicing returns a sequence:

\begin{verbatim}
In [68]: l
Out[68]: [0, 1, 2, 3, 4, 5, 6, 7, 8, 9]

In [69]: l[2:4]
Out[69]: [2, 3]
\end{verbatim}

Even if it's one element long

\begin{verbatim}
In [70]: l[2:3]
Out[70]: [2]

In [71]: type(l[2:3])
Out[71]: list
\end{verbatim}

\end{frame}

%-------------------------------
\begin{frame}[fragile]{Slicing vs. Indexing}

{\Large Indexing out of range produces an error}
\vfill
\begin{verbatim}
In [129]: s = "a bunch of words"
In [130]: s[17]
----> 1 s[17]
IndexError: string index out of range
\end{verbatim}

\vfill
{\Large Slicing just gives you what's there}

\begin{verbatim}
In [131]: s[10:20]
Out[131]: ' words'

In [132]: s[20:30]
Out[132]: ''
\end{verbatim}
(demo)
\end{frame}

% ---------------------------------------------
\begin{frame}[fragile]{Multiplying and slicing}

{\Large from CodingBat: Warmup-1 -- front3}

\begin{verbatim}
def front3(str):
  if len(str) < 3:
    return str+str+str
  else:
    return str[:3]+str[:3]+str[:3]
\end{verbatim}

{\Large or}

\begin{verbatim}
def front3(str):
    return str[:3] * 3
\end{verbatim}

\end{frame} 

% ---------------------------------------------
\begin{frame}[fragile]{Slicing}

{\Large from CodingBat: Warmup-1 -- \verb+missing_char+ }

\begin{verbatim}
def missing_char(str, n):
  front = str[0:n]
  l = len(str)-1
  back = str[n+1:l+1]
  return front + back
\end{verbatim}

\begin{verbatim}
def missing_char(str, n):
    return str[:n] + str[n+1:]
\end{verbatim}

\end{frame} 

% ---------------------------------------------
\begin{frame}[fragile]{Slicing}

{\Large you can skip items, too}

\begin{verbatim}
In [289]: string = "a fairly long string"

In [290]: string[0:15]
Out[290]: 'a fairly long s'

In [291]: string[0:15:2]
Out[291]: 'afil ogs'

In [292]: string[0:15:3]
Out[292]: 'aallg'
\end{verbatim}

\end{frame} 


%-------------------------------
\begin{frame}{LAB}
Write some functions that:
\begin{itemize}
\item return a string with the first and last characters exchanged.
\item return a string with every other character removed
\item return a string with the first and last 4 characters removed, and every other char in between
\item return a string reversed (just with slicing)
\item return a string with the middle, then last, then first third in a new order
\end{itemize}
\end{frame}

%-------------------------------
\begin{frame}{Lightning Talk}

{\LARGE Lightning Talk: Myke }

\end{frame}

\section{Looping}

%-------------------------------
\begin{frame}[fragile]{for loops}

{\Large looping through sequences

\begin{verbatim}
for x in sequence:
    do_something_with_x
\end{verbatim}
}
\end{frame}

%-------------------------------
\begin{frame}[fragile]{for loops}

\begin{verbatim}
In [170]: for x in "a string":
   .....:         print x
   .....:     
a
 
s
t
r
i
n
g
\end{verbatim}
\end{frame}

%-------------------------------
\begin{frame}[fragile]{range}

{\Large looping a known number of times..}

\begin{verbatim}
In [171]: for i in range(5):
   .....:     print i
   .....:     
0
1
2
3
4
\end{verbatim}
(you don't need to do anything with i...
\end{frame}

%-------------------------------
\begin{frame}[fragile]{range}

{\Large \verb|range| defined similarly to indexing}

\begin{verbatim}
In [183]: range(4)
Out[183]: [0, 1, 2, 3]

In [184]: range(2,4)
Out[184]: [2, 3]

In [185]: range(2,10,2)
Out[185]: [2, 4, 6, 8]
\end{verbatim}

\end{frame}

%-------------------------------
\begin{frame}[fragile]{indexing?}

{\Large Python only loops through a sequence -- not like C, Javascript, etc...}
\begin{verbatim}
for(var i=0; i<arr.length; i++) {
    var value = arr[i];
    alert(i =") "+value);
}
\end{verbatim}

\end{frame}

%-------------------------------
\begin{frame}[fragile]{indexing?}

{\Large Use range?}
\begin{verbatim}
In [193]: letters = "Python"

In [194]: for i in range(len(letters)):
   .....:     print letters[i]
   .....:     
P
y
t
h
o
n
\end{verbatim}

\end{frame}

%-------------------------------
\begin{frame}[fragile]{indexing?}

{\Large More Pythonic -- for loops through sequences}
\begin{verbatim}
In [196]: for l in letters:
   .....:     print l
   .....:     
P
y
t
h
o
n
\end{verbatim}
\vfill
{\large Never index in normal cases}
\end{frame}

%-------------------------------
\begin{frame}[fragile]{enumerate}

{\Large If you need an index -- \verb|enumerate|}
\begin{verbatim}
In [197]: for i, l in enumerate(letters):
   .....:     print i, l
   .....:     
0 P
1 y
2 t
3 h
4 o
5 n
\end{verbatim}
\end{frame}

%-------------------------------
\begin{frame}[fragile]{multiple sequences -- zip}

{\Large If you need to loop though parallel sequences -- \verb|zip|}
\begin{verbatim}
In [200]: first_names = ['Fred', 'Mary', 'Jane']

In [201]: last_names = ['Baker', 'Jones', 'Miller']

In [203]: for first, last in zip(first_names, last_names):
   .....:     print first, last
   .....:     
Fred Baker
Mary Jones
Jane Miller
\end{verbatim}
\end{frame}

%-------------------------------
\begin{frame}[fragile]{xrange}

{\Large \verb|range| creates the whole list}

\vfill
{\Large \verb|xrange| is a generator -- creates it as it's needed --}

\vfill
{\Large a good idea for large numbers}

\begin{verbatim}
In [207]: for i in xrange(3):
   .....:     print i
0
1
2
\end{verbatim}
(Python 3 -- \verb|range == xrange|)
\end{frame}



%-------------------------------
\begin{frame}[fragile]{for}

{\Large \verb|for| does NOT create a name space:}

\begin{verbatim}
In [172]: x = 10

In [173]: for x in range(3):
   .....:     pass
   .....: 

In [174]: x
Out[174]: 2
\end{verbatim}
\end{frame}

%-------------------------------
\begin{frame}[fragile]{LAB}

\begin{itemize}

  \item Look up the \verb+%+ operator. What do these do?

    \verb| 10 % 7 == 3 |

    \verb| 14 % 7 == 0 |

  \item  Write a program that prints the numbers from 1 to 100 inclusive.
But for multiples of three print ``Fizz'' instead of the number and for the
multiples of five print ``Buzz''. For numbers which are multiples of both three
and five print ``FizzBuzz'' instead.

\end{itemize}

\end{frame}


%-------------------------------
\begin{frame}[fragile]{while}

{\Large \verb|while| is for when you don't know how many loops you need}

\vfill
{\Large Continues to execute the body until condition is not \verb|True|}

\begin{verbatim}
while a_condition:
   some_code
   in_the_body
\end{verbatim}
\end{frame}



%-------------------------------
\begin{frame}[fragile]{while}

{\Large \verb|while| is more general than \verb|for| -- 
you can always express for as while,
but not always vice-versa.}

\vfill

{\Large \verb|while| is more error-prone -- requires some care to terminate}

\vfill
{\Large  loop body must make progress, so condition can become \verb|False| }

\vfill
{\Large  potential error: infinite loops }
\end{frame}



%-------------------------------
\begin{frame}[fragile]{while vs. for}

\begin{verbatim}
letters = 'Python'
i=0
while i < len(letters):
    print letters[i]
    i += 1
\end{verbatim}
vs.
\begin{verbatim}
letters = 'Python'
for c in letters:
    print c
\end{verbatim}

\end{frame}


%-------------------------------
\begin{frame}[fragile]{while}

{\Large Shortcut: recall -- 0 or empty sequence is \verb|False| }

%\begin{verbatim}
%while x:    # terminates if x >= 0 on entry
%    ...     # do something with x
%    x -= 1  # make progress toward 0
%\end{verbatim}

\end{frame}


%-------------------------------
\begin{frame}[fragile]{break}

{\Large \verb|break| ends a loop early}

\begin{verbatim}
x = 0
while True:
    print x
    if x > 3:
        break
    x = x + 1
In [216]: run for_while.py
0
1
2
3
4
\end{verbatim}

This is a pretty common idiom

\end{frame}

%-------------------------------
\begin{frame}[fragile]{break}

{\Large same way with a \verb|for| loop }

\begin{verbatim}
name = "Chris Barker"
for c in name:
    print c,
    if c == "B":
        break
print "I'm done"

C h r i s   B 
I'm done
\end{verbatim}
\end{frame}

\begin{frame}[fragile]{continue}

{\Large \verb|continue| skips to the start of the loop again}

\begin{verbatim}
print "continue in a for loop"
name = "Chris Barker"
for c in name:
    if c == "B":
        continue
    print c,
print "\nI'm done"

continue in a for loop
C h r i s   a r k e r 
I'm done
\end{verbatim}
\end{frame}

\begin{frame}[fragile]{continue}

{\Large \verb|continue| works for a \verb|while| loop too.}

\begin{verbatim}
print "continue in a while loop"
x = 6
while x > 0:
    x = x-1
    if x%2:
        continue
    print x,
print "\nI'm done"

continue in a while loop
4 2 0 
I'm done
\end{verbatim}
\end{frame}

\begin{frame}[fragile]{else again}

{\Large \verb|else| block run if the loop finished naturally -- no \verb|break|}

\begin{verbatim}
print "else in a for loop"
x = 5
for i in range(5):
    print i
    if i == x:
        break
else:
    print "else block run"

\end{verbatim}
\end{frame}

% ---------------------------------------
\begin{frame}[fragile]{Command Line Input}

{\Large \verb|input| and \verb|raw_input|}

{\Large \verb|input| evaluates the input}
\begin{verbatim}
In [265]: val = input("a message> ")
a message> 4.5

In [266]: type(val)
Out[266]: float
\end{verbatim}

{\Large \verb|raw_input| gives you the plain string}
\begin{verbatim}
In [265]: val = input("a message> ")
a message> 4.5

In [266]: type(val)
Out[266]: float
\end{verbatim}
(demo)
\end{frame}


%-------------------------------
\begin{frame}[fragile]{LAB}

\verb| def count_them(letter): |
\begin{itemize}
  \item prompts the user to input a letter
  \item counts the number of times the given letter is input
  \item prompts the user for another letter
  \item continues until the user inputs "x"
  \item returns the count of the letter input
\end{itemize}

\verb| def count_letter_in_string(string, letter): |
\begin{itemize}
  \item counts the number of instances of the letter in the string
  \item ends when a period is encountered
  \item if no period is encountered -- prints "hey, there was no period!"
\end{itemize}
\end{frame}

%-------------------------------
\begin{frame}[fragile]{example}

\Large{Example: \verb|re-run-latex.py| script}

\end{frame}


%-------------------------------
\begin{frame}[fragile]{Homework}

Read:
\begin{itemize}
  \item Read TP: 9, 14
  \item extra: string methods: \url{http://docs.python.org/library/stdtypes.html#string-methods}
  \item extra: unicode: \url{http://www.joelonsoftware.com/articles/Unicode.html}
\end{itemize}

Do:
\begin{itemize}
    \item Six more CodingBat exercises. 
    \item LPTHW: for extra practice with the concepts -- some of:
    \begin{description}
        \item[strings:] ex5, ex6, ex7, ex8, ex9, ex10
        \item[raw\_input(), sys.argv:] ex12, ex13, ex14 (needed for files)
        \item[files:] ex15, ex16, ex17 
    \end{description}    
\end{itemize}

\end{frame}

\end{document}

 
