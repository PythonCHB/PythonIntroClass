\documentclass{beamer}
%\usepackage[latin1]{inputenc}
\usetheme{Warsaw}
\title[Week 5: Building a Web Server from Scratch]{Introduction to Python}
\author{Christopher Barker}
\institute{UW Continuing Education / Isilon}
\date{July 25, 2012}

\usepackage{listings}
\usepackage{hyperref}

\begin{document}

% ---------------------------------------------
\begin{frame}
  \titlepage
\end{frame}

% ---------------------------------------------
\begin{frame}
\frametitle{Table of Contents}
%\tableofcontents[currentsection]
  \tableofcontents

\vfill
Lightning Talks: David, Drew and Phillip
\end{frame}


\section{Review/Questions}


% ---------------------------------------------
\begin{frame}{"Grading"}

{\Large To pass this class, you need to:}

\begin{itemize}
  \item Come to most of the classes
  \item Be engaged when you are here
  \item Complete a project:
  \begin{itemize}
      \item Modest size (two weeks of spare time)
      \item Something useful (or fun) to you.
      \item Demonstrates that you've got (at least) a basic grasp of Python.
  \end{itemize}
\end{itemize}

\end{frame}


% ---------------------------------------------
\begin{frame}{Review of Previous Class}

\begin{itemize}
  \item Keyword arguments/parameters
  \item Lists
  \item Dictionaries
  \item Sets
\end{itemize}

\end{frame}

% ---------------------------------------------
\begin{frame}[fragile]{Homework review}

{\Large Dave Thomas Coding Kata 14 } 

\vfill
\url{http://codekata.pragprog.com/2007/01/kata_fourteen_t.html}

\vfill
My results: \\
honour my gems hundred year old forever do not even two
might end was a he realised how yes when i stronger if i perfectly happy and
revealed it to windows were blocked a thing is to sound him rd floor that would
induce the work by people must sit without with jabez wilson bad compliment
...

\vfill
My code: \verb+week-04/code/trigram.py+

\end{frame}

%-------------------------------
\begin{frame}{Lightning Talk}

{\centering

\vfill
{\LARGE Lightning Talk:  }

\vfill
{\Huge David}

\vfill
}
\end{frame}


\begin{frame}{Class Structure}

{\Large
This class is different -- more a tutorial than a class: lots of coding.

\vfill
We're going to run through building a really basic HTTP server from the ground up.

\vfill
We'll see how far we get.
}

\vfill
Note: I'm no expert -- I'm learning along with you...
\end{frame}

\section{Sockets}

% ---------------------------------------------
\begin{frame}[fragile]{Sockets}

{\Large ``Socket'' at either end of a pathway: client and server can be
"plugged in" to communicate}

\vfill
{\Large Five pieces of data to uniquely identify a connection:}

\begin{itemize}
  \item Transport protocol (UDP, TCP) (we'll use TCP)
  \item Remote IP address
  \item Remote port number
  \item Local IP address
  \item Local port number 
\end{itemize}

\vfill
(use localhost (127.0.0.1) on both ends for this class...)
\end{frame} 

%-------------------------------
\begin{frame}[fragile]{Python Socket Module}

{\Large Create a socket:}
\begin{verbatim}
s = socket.socket(socket.AF_INET, socket.SOCK_STREAM)
\end{verbatim}
\verb| AF_INET | : Internet Family of Protocols

\verb| SOCK_STREAM | : TCP

\vfill
{\Large Set an option:}
\begin{verbatim}
s.setsockopt(socket.SOL_SOCKET, socket.SO_REUSEADDR, 1)  
\end{verbatim}
 re-use the address -- so the OS won't reserve it

\vfill
(Python docs say "see the UNIX man pages...)
\end{frame}

%-------------------------------
\begin{frame}[fragile]{A socket server}

{\Large \verb|echo_server.py|}
\begin{verbatim}
s = socket.socket(socket.AF_INET, socket.SOCK_STREAM) 

# the bind makes it a server
s.bind( (host, port) ) 
s.listen(backlog) 

while True: # keep looking for new connections forever
    client, address = s.accept() # look for a connection
    data = client.recv(size)
    if data: # if the connection was closed there would be no data
        print "received: %s, sending it back"%data
        client.send(data) 
        client.close()
\end{verbatim}
\end{frame}

%-------------------------------
\begin{frame}[fragile]{A socket client}

{\Large \verb|echo_client.py|}
\begin{verbatim}
while True:
    s = socket.socket(socket.AF_INET, 
                      socket.SOCK_STREAM) 
    s.connect((host,port)) 
    msg = raw_input('what should I send? >> ')
    if msg:
        s.send(msg) 
        data = s.recv(size) 
        s.close() 
\end{verbatim}
\end{frame}

%-------------------------------
\begin{frame}[fragile]{Mini-LAB}

{\Large Start up \verb|echo_server.py|}

\vfill
{\Large Start up \verb|echo_client.py|}

\vfill
(in different terminals...)

\vfill
{\Large Watch what happens when you use the client}

\vfill
If any of you are using a shared system -- change your port numbers

NOTE: running from iPython can cause trouble...
\end{frame}


%-------------------------------
\begin{frame}{Lightning Talk}

{\centering

\vfill
{\LARGE Lightning Talk:  }

\vfill
{\Huge Drew}

\vfill
}
\end{frame}


\section{HTTP}

% ---------------------------------------------
\begin{frame}[fragile]{HTTP}

\centering{ \LARGE {\bf H}yper{\bf T}ext {\bf T}ransfer {\bf P}rotocol } 
\vfill
{\Large Client-Server: }
\begin{itemize}
  \item requests
  \item responses
\end{itemize}

{\Large Each has:}

\begin{itemize}
  \item Method specification (request)
  \item Status line (response)
  \item  Headers (RFC 822-compliant)\\[0.1in]
  (optionally)
  \item Entity headers and body
\end{itemize}
 
\vfill
 ( RFC 2616 )

\end{frame} 

%-------------------------------
\begin{frame}{HTTP Requests}

{\Large Request Methods  }

\begin{itemize}
  \item GET  -- Read URI content
  \item HEAD -- GET headers only
  \item POST -- Create
  \item PUT  -- Update (entity transfer to server)
  \item DELETE  -- delete content
\end{itemize}
(GET and POST different ways to do similar things...)

\vfill
There are four others -- but these are the ones most used
\end{frame}

%-------------------------------
\begin{frame}[fragile]{HTTP request}

{\Large Example HTTP GET request}
\begin{verbatim}
GET /a_file HTTP/1.1
Host: localhost:55555
User-Agent: Mozilla/5.0 (Macintosh; Intel Mac OS X 10.7; rv:12.0) Gecko/20100101 Firefox/12.0
Accept: text/html,application/xhtml+xml,application/xml;q=0.9,*/*;q=0.8
Accept-Language: en-us,en;q=0.5
Accept-Encoding: gzip, deflate
Connection: keep-alive
\end{verbatim}

\end{frame}

%-------------------------------
\begin{frame}{HTTP Responses}

{\Large Response Codes}

\begin{itemize}
\item 200 OK
\item 404 Not Found
\item 301 Moved Permanently
\item 302 Moved Temporarily
\item 303 See Other (HTTP 1.1 only)
\item 500 Server Error
\end{itemize}

\vfill
There are four others -- but these are the ones most used
\end{frame}

%-------------------------------
\begin{frame}[fragile]{HTTP Response header}

\begin{verbatim}
HTTP/1.1 200 OK
Date: Fri, 31 Dec 1999 23:59:59 GMT
Content-Type: text/html
Content-Length: 1354

<html>
<body>
<h1>Happy New Millennium!</h1>
(more file contents)
... </body> </html>
\end{verbatim}

\vfill
Blank line between header and body critical!\\
 \hspace{0.25in} (\verb|\r\n| linefeeds)
\end{frame}

%-------------------------------
\begin{frame}[fragile]{HTTP Response header}

{\Large Header-Name: value}

\vfill
{\Large Quick reference to HTTP headers:}

\vfill
\url{http://www.cs.tut.fi/~jkorpela/http.html}
\end{frame}

%-------------------------------
\begin{frame}[fragile]{HTTP Response header}

{\Large body data:

\vfill
\verb| Content-Type: xyz/abc | 

\vfill
Mime types we might want:
}

\begin{itemize}
  \item \verb|text/plain|
  \item \verb|text/html|
  \item \verb|image/png|
  \item \verb|image/jpeg|
\end{itemize}

\vfill
\url{http://www.webmaster-toolkit.com/mime-types.shtml}
\end{frame}

%-------------------------------
\begin{frame}[fragile]{Debugging}

{\LargeDebugging Tools}

\begin{itemize}
  \item windows:\\
    \url{http://www.fiddler2.com/fiddler2/}
  \item windows \& mac:\\
    \url{http://www.charlesproxy.com/}
  \item Firefox:\\ 
    \url{http://getfirebug.com/}
  \item Safari, Chrome and IE: built in
\end{itemize}

\end{frame}

\section{Building an HTTP server}

%%-------------------------------
\begin{frame}[fragile]{Building an HTTP Server}

\vfill
{\Large We've got everything we need to know to build a simple server}

\vfill
(GET only for now...)

\vfill
{\LARGE Build an HTTP server that can serve up the files in:
\verb|week-05\code\web|}\\

\end{frame}



%-------------------------------
\begin{frame}{Building an HTTP Server}

{\Large Incremental Development:}
\begin{enumerate}
  \item A socket server that can receive a request (and print that request to the console)
  \item Server returns a simple reply
  \item Server returns a properly formatted HTTP reply
  \item Server returns a 404 error
  \item Server returns the file asked for
  \item Server returns a directory listing
  \item Server returns multiple file types
  \item Server returns a calculated response
\end{enumerate}

\end{frame}

%-------------------------------
\begin{frame}[fragile]{http\_serve1.py}

{\Large Edit \verb|echo_serve1.py| to print the request:}
\begin{enumerate}
  \item Point your browser at \verb|echo_server.py| -- what do you get?
  \item Save it as \verb| http_serve1.py |
  \item Edit it to print the request to the console
  \item Edit it to return a bit of html (\verb| tiny_html.html |)
  \item What happens when you point your browser to it? \\
        Try a couple different browsers -- I get a different result with Firefox and Safari
\end{enumerate}

\end{frame}

%-------------------------------
\begin{frame}[fragile]{http\_serve2.py}

{\Large Return a properly formatted HTTP response:}
\begin{enumerate}
  \item Save \verb|http_serve1.py|  as \verb|http_serve2.py|
  \item Add code that generates an HTTP ``200 OK'' header \\
        (don't forget the blank line! (\verb|\r\n|)
  \item Use \verb|httpdate.httpdate_now()| to give you an HTTP date string
  \item What happens when you point your browser to it now?
\end{enumerate}

\end{frame}

%-------------------------------
\begin{frame}{Lightning Talk}

{\centering

\vfill
{\LARGE Lightning Talk:  }

\vfill
{\Huge Phillip}

\vfill
}
\end{frame}


%-------------------------------
\begin{frame}[fragile]{http\_serve3.py}

{\Large Parse the request:}
\begin{enumerate}
  \item Save \verb|http_serve2.py|  as \verb|http_serve3.py|
  \item Add code that parses the HTTP request -- it should give you the URI requested
  \item Have it check to make sure it's a GET request
  \item print the URI (file name) to the console
\end{enumerate}

\end{frame}

%-------------------------------
\begin{frame}[fragile]{http\_serve4.py}

{\Large Return a Listing:}
\begin{enumerate}
  \item Save \verb|http_serve3.py|  as \verb|http_serve4.py|
  \item Add code that parses the URI -- so you can figure out what file is requested
  \item check to see if is a directory or a file
  \item return a listing (simple text) of the dir if it's a dir
  \item return a 404 otherwise
\end{enumerate}

\vfill
Try: \\
\url{http://localhost:50000/images}
\end{frame}

%-------------------------------
\begin{frame}[fragile]{http\_serve5.py}

{\Large Support various file types:}
\begin{enumerate}
  \item Save \verb|http_serve4.py|  as \verb|http_serve5.py|
  \item If the request is for a file -- return that file
  \item have it be a different mime type depending on the type of file
  \item support: \verb| .html, .txt, .jpeg, .png|
\end{enumerate}

\vfill
Try: \\
\url{http://localhost:50000/sample.txt}
\url{http://localhost:50000/images/sample_1.png}

\vfill
{\Large You now have a pretty functional web server!} 
\end{frame}

%-------------------------------
\begin{frame}[fragile]{http\_serve6.py}

{\Large If we have time...}
\begin{enumerate}
  \item Save \verb|http_serve5.py|  as \verb|http_serve6.py|
  \item Format the Dir listing as HTML
  \item Make the files in the listing clickable links.
\end{enumerate}
 and / or
\begin{enumerate}
  \item Make a simple web app (non-static)
  \item Have \url{localhost:50000/the_time} return a web page with the current time.
\end{enumerate}
\vfill
{\Large You now have a very functional web server!} 
\end{frame}

\section{Wrap Up / Homework}

%-------------------------------
\begin{frame}[fragile]{Standard Library Support}

{\Large It's unlikely that you'll need to use raw sockets}

\vfill
{\Large Standard Library Modules}
\begin{itemize}
  \item httplib
  \item urllib2 (requests (PyPI) )
  \item smtplib
  \item poplib
  \item imaplib
\end{itemize}

{\Large Third Party Modules}
\begin{itemize}
  \item requests (cleaner interface that urllib2)
  \item paramiko (SSH)
  \item Probably one for anything you're likely to do...
\end{itemize}
  
\end{frame}

%-------------------------------
\begin{frame}[fragile]{Homework}

\begin{itemize}
  \item Think Python ch. 15-18
  \item Finish what you didn't get to in class
  \item When a *.py file is asked for: 
    \begin{itemize}
      \item Don't return the script's contents
      \item Return the result of running the script (stdout)\\
            (\verb|subprocess.Popen()| will be useful)
      \item Test with a script that prints the time (or something...)
      \item You've just re-invented CGI
    \end{itemize}
\end{itemize}
\end{frame}

\end{document}

 
